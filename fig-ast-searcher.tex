\newcounter{asdasd}

\begin{figure}[ht]
\centering
\begin{tikzpicture}[
label distance=-10pt,
every label/.style={blue},
lbl/.style={label=160:{\stepcounter{asdasd}\arabic{asdasd}}}
]
  \node at (0, 0) {\mi{constraint}} [sibling distance=60pt]
  child {node[lbl] {\cpp{BinOp} =}
    child {node[lbl] {\cpp{BinOp} +}
      child {node[lbl] (X) {\cpp{Id} ``x''}}
      child {node[lbl] {\cpp{IntLit} 42}}
    }
    child {node[lbl] {\cpp{Id} ``y''}}
  };

  \node (VX) at (6.5, 0) {\mi{var int: x}} [level 2/.style={sibling distance=140pt}, level 3/.style={sibling distance=65pt}]
  child {node[lbl] {\cpp{BinOp} +}
    child {node[lbl] {\cpp{BinOp} +}
      child {node[lbl] {\cpp{IntLit} 1}}
      child {node[lbl] {\cpp{IntLit} 1}}
    }
    child {node[label=177:{\stepcounter{asdasd}\arabic{asdasd}}] {\cpp{BinOp} +}
      child {node[lbl] {\cpp{IntLit} 1}}
      child {node[lbl] {\cpp{IntLit} 1}}
    }
  };

  \draw[-{Straight Barb[scale=1.3]},dashed] plot [smooth] coordinates {(X.south) ($(X.south) + (3, -0.3)$) ($(VX.west) + (-1.2, -0.5)$) (VX.west)};
\end{tikzpicture}
\caption{Illustration of two ASTs. The left one corresponds to \mi{constraint x+42=y},
  and the right one corresponds to \mi{var int: x=1+1+1+1}. The blue numbers are
  for referring to the nodes more easily and are not a part of the AST. The \mi{Id} at
  node 3 has a pointer to its declaration. \cpp{BinOp} is short for ``binary operation'',
  \cpp{Id} is short for ``identifier'' and lastly \cpp{IntLit} is short for ``integer
  literal''. These three are names of classes from MiniZinc.}%
\label{fig:ast:searcher}
\end{figure}
