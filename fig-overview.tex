\begin{figure}[ht]
\centering
\smallskip%
\begin{tikzpicture}[
  blackrekt/.style={draw, rectangle},
  number/.style={auto,draw,circle,solid,inner sep=1pt,outer sep=3pt, node font=\scriptsize, very thin},
]
  \node (LE) at (0, 0) {\cpp{LintEnv}};
  \node (IP) [blackrekt, below=0.2 of LE.south west, anchor=north west] {Include paths};
  \node (AST) [blackrekt, right=0.2 of IP] {AST};
  \node (RES) [double copy shadow={shadow xshift=4pt, shadow yshift=0}, fill=white, blackrekt, below=0.2 of IP.south west, anchor=north west] {\cpp{LintResult}};
  \node (BOX) [fit=(LE) (IP) (AST) (RES), blackrekt] {};

  \newcommand{\lintrule}[2]{
    \node (#1) at #2 {\cpp{LintRule}};
    \draw[fill=white] (#1.north west) +(0, 0.5) -| ($(#1.north west)!0.3!(#1.north east)$)
    -- ($(#1.north west)!0.7!(#1.north east)$) |- ($(#1.north east) + (0, 0.5)$)
    -- (#1.south east) -- (#1.south west) -- cycle;
  }
  \foreach \y in {-4.2, -4.1, -4} \lintrule{LR1}{(-\y, \y)};
  \node (PR1) at (4, -4) {\cpp{LintRule}};
  \draw[big arrow, shorten >=8pt] (BOX) edge[out=0, in=90] node [number, near start] {1} (LR1);
  \draw[-{Tee Barb[width=5mm, length=2mm]}, shorten >=0, dashed] (LR1) edge[out=180,
  in=-90] node [number] {2} ($(BOX.south) + (0.5, 0)$);

  \node (PRT) [rectangle, align=left, draw=black, line width=1mm, rounded corners, inner sep=2mm] at (-3, -3.6) {\texttt{\textbf{\Large >\_}} \\ Stdout printer};
  \draw [big arrow] (RES) edge[in=90,out=180,looseness=1.3] node [number] {3} (PRT);

\end{tikzpicture}%
\smallskip%
\caption{Illustration for the main execution of the linter. The \cpp{LintEnv} is given as an argument to each \cpp{LintRule} (1). Each rule will analyse the AST and write its results back to a list inside \cpp{LintEnv} (2). When all rules have been processed is the list of all results given to a function that will output them back to the user (3).}%
\label{fig:overview}
\end{figure}
