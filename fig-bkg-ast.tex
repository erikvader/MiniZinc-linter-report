\begin{figure}[ht]
\centering
\begin{tikzpicture}[AST, eval/.style={red, font=\footnotesize}]
  \node (l) at (0, 0) {*}
    child {node {0}}
    child {node {+}
      child {node {2}}
      child {node {$f$}
        child {node {3}}}};

  \node [eval, above right=-6pt and 0pt of l-2-2-1] {3};
  \node [eval, above right=-6pt and 0pt of l-2-2] {3};
  \node [eval, above right=-6pt and 0pt of l-2-1] {2};
  \node [eval, above right=-6pt and 0pt of l-2] {5};
  \node [eval, above right=-6pt and 0pt of l-1] {0};
  \node [eval, above right=-6pt and 0pt of l] {0};

  \node (r) at (6, 0) {+}
    child {node {*}
      child {node {0}}
      child {node {2}}}
    child {node {$f$}
      child {node {3}}};

  \node [eval, above right=-6pt and 0pt of r-2-1] {3};
  \node [eval, above right=-6pt and 0pt of r-2] {3};
  \node [eval, above right=-6pt and 0pt of r-1-1] {0};
  \node [eval, above right=-6pt and 0pt of r-1-2] {2};
  \node [eval, above right=-6pt and 0pt of r-1] {0};
  \node [eval, above right=-6pt and 0pt of r] {3};
\end{tikzpicture}
\caption{On the left is a possible tree of the expression $0\cdot(2+f(3))$. The right one
  is a possible tree of the expression $(0 \cdot 2)+f(3)$. The asterisks in the trees
  represents multiplication. The different placements of the parenthesis changes what the
  generated tree looks like. The red numbers on the sides represents what each sub-tree
  evaluates to if $f(x) = x$.}%
\label{fig:bkg:ast}
\end{figure}
