\begin{figure}[ht]
  \centering
  \begin{tikzpicture}[
    city/.style={draw, circle, minimum size=9mm},
    arr/.style={rectangle, minimum size=1cm},
    ]
    \node (SLA) at (-3, -1) [city, label=above:Sala] {1};
    \node (HBY) at (2, 0) [city, label=above:Heby] {2};
    \node (MGA) at (5, -0.2) [city, label=above:Morgongåva] {3};
    \node (VIT) at (7.3, -1.4) [city, label=above:Vittinge] {4};

    \draw (SLA) edge[above, near start] node {20} (MGA)
          (MGA) edge[above] node {5} (HBY)
          (HBY) edge[above, near end] node {12} (VIT)
          (VIT) edge[below] node {26} (SLA);

    \node (A1) at (0, -3) [arr] {1};
    \node (A2) [arr, right=0 of A1] {3};
    \node (A3) [arr, right=0 of A2] {2};
    \node (A4) [arr, right=0 of A3] {4};
    \draw (A1.north west) -- (A4.north east) -- (A4.south east) -- (A1.south west) -- cycle;
    \draw (A1.north east) -- (A1.south east)
          (A2.north east) -- (A2.south east)
          (A3.north east) -- (A3.south east);

    \node [right=of A4] {$\sum = \SI{63}{km}$};
  \end{tikzpicture}
  \caption{Example instance of Travelling Salesman Problem with $n=4$ cities. The state is
    $\langle 1,3,2,4 \rangle$, which means the route starts at city 1, proceeding to city
    3, then 2 and lastly 4 and back. The total distance travelled is \SI{63}{km}.}%
  \label{fig:tsp}
\end{figure}
