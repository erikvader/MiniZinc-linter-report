\begin{figure}[ht]
  \centering
  \begin{tikzpicture}
    \begin{axis} [
      xmode=log,
      % ymode=log,
      no markers,
      grid=major,
      ymin=0,
      enlargelimits=false,
      title=AST sizes,
      xlabel=Number of nodes,
      ylabel=Counts,
      width=0.45\textwidth,
      tick pos=left,
      ]
      \addplot table [x=numnodes, y=occur, col sep=comma] {counts.csv};
    \end{axis}
  \end{tikzpicture}\hfill%
  \begin{tikzpicture}
    \begin{axis}[
      no markers,
      xmode=log,
      enlargelimits=false,
      ymin=0,
      ymax=1.1,
      ytick={1,0.9,0.5},
      grid=major,
      log ticks with fixed point,
      xtick={1,14,60,1000,917442},
      title={Normalised Cumulative Distribution},
      xlabel=Number of nodes,
      ylabel=Percentage,
      width=0.45\textwidth,
      tick pos=left,
      ]
      \addplot table [x=numnodes, y=perc, col sep=comma] {counts.csv};
    \end{axis}
  \end{tikzpicture}
  \caption{The AST sizes (number of nodes they consist of) on all models in the MiniZinc
    Benchmarks. An AST in this case is a top-level item, not a whole model. The left graph
    show how many times various AST sizes occur, and the right graph show a scaled
    cumulative sum of the left one. The largest AST had a size of 917442, and occurred 4
    times. Most ASTs were small, 92\% of all ASTs consisted of 14 nodes or less, and
    99.5\% consisted of 60 nodes or less. These numbers come from running the AST
    searcher, configured to match a single node of any type anywhere in the tree, and
    counting the number of matches.}%
  \label{fig:ast:counts}
\end{figure}
